\usepackage{hyperref}

\documentclass{article}

\begin{document}

\section*{Notes}

\begin{itemize}
\item Installation de Virtual Box (déjà fait)
\item Téléchargement de Debian (64 bits)
\item Installation de Debian 64 bits après avoir activé Intel VMX dans le Bios (pas trouvé mais ça a quand même marché, je l’ai sûrement fait il y a longtemps)
\item Fin de l’installation de Debian
\item Installation de Visual Code Studio
\item Installation de Python (depuis VS code)
\item Installation de PHP et MySQL (depuis VS code)
\item Utilisation des ressources de:
\begin{itemize}
\item \textit{https://openclassrooms.com/fr/courses/918836-concevez-votre-site-web-avec-php-et-mysql}
\item \textit{https://openclassrooms.com/fr/courses/1603881-creez-votre-site-web-avec-html5-et-css3}
\end{itemize}
\item Installation de l'extension LaTex de VS Code
\item INFO: VS CODE est open source, gratuit et développé par Microsoft
\item Installation de Open Browser Preview pour VS Code
\item Installation de git pour Debian (afin de télécharger des fichiers depuis Github)
\item Installation de PHP depuis le terminal (\texttt{sudo apt install php})
\item Installation et configuration de Apache2 grâce à \textit{https://www.youtube.com/watch?v=1CDxpAzvLKY}
\item Installation de MySQL pour Debian 11 depuis internet
\item Installation de php, phpmadmin et php-mysql
\item Recherches sur internet (\textit{"How To Create A MySQL Database In phpMyAdmin"}) de comment créer une base de donnée et une table
\item Création des tables de la base de donnée "siteweb"
\item Création des pages: \texttt{index.php}, \texttt{login.php}, \texttt{logout.php}, \texttt{signup.php}, \texttt{account.php}, \texttt{connexion.php}, \texttt{functions.php} et \texttt{style.css}
\end{itemize}

\end{document}